\documentclass{article}[12pt]

\usepackage{amsmath}
\usepackage{amssymb}
\usepackage{amsthm}
\usepackage{array}
\usepackage{xy}
\usepackage{fancyhdr}
\usepackage[pdftex]{graphicx}
\usepackage{setspace}
\usepackage{tikz}
\usepackage{float}
\usepackage{color}
\usepackage{thmtools}
\usepackage{arcs}
\usepackage{enumitem}
\usepackage{asymptote}
\usepackage{hyperref}
\usepackage{etoolbox}

\makeatletter
\newcommand\myemail[3]{%                %\newcommand\tpj@compose@mailto[3]{%
\edef\@tempa{mailto:#1?subject=#2 }%
\edef\@tempb{\expandafter\html@spaces\@tempa\@empty}%
\href{\@tempb}{#3}}

\catcode\%=11
\def\html@spaces#1 #2{#1%20\ifx#2\@empty\else\expandafter\html@spaces\fi#2}
\catcode\%=14
\makeatother

\pdfpagewidth 8.5in
\pdfpageheight 11in
\oddsidemargin -0.25in
\textwidth 7in
\textheight 9in
\topmargin -1.8in
\headheight 112pt

\pagestyle{fancy}
\doublespacing

\title{Random Walk\\Construction Plan}
\author{Garrett Higginbotham}

\lhead{Random Walk}
\rhead{Construction Plan}

\lfoot{Property of Garrett Higginbotham.\\Please send any suggestions to \myemail{ghiggie@uab.edu}{Document Suggestions-Random Walk, Construction Plans}{ghiggie@uab.edu}}
\rfoot{Last Modified: \today}

\begin{document}

\centerline{\Large{Random Walk}}
\centerline{\Large{Construction Plan}}

\vskip 0.5in

Currently, I am constructing a model of a random walk within a region with other static objects. Below, you will find a plan of how I will proceed with constructing the model.

\begin{description}
	\item[Version 1] In the initial commit, a model was built to describe a particle beginning at a random location within a region. At each time step, a random direction is selected to describe the direction the particle will take.
	\item[Version 2] In Version 1, the particle was not constrained to remain in the region, which is a physically unrealistic system. This version removes this issue.
	\item[Version 3] The previous versions simply displayed the coordinates of the particle at each point in time. This version will provide an animation to illustrate the motion of the particle.
	\item[Version 4] In this version, the model becomes more complicated. I will add an object to the region. If at any time the particle makes contact with the object, it will stick and the program will terminate.
	\item[Version 5] This version will extend the number of dimensions to 3, and will add in the ability for the user to adjust the number of particles in the region.
	\item[Version 6] In Version 5, the capability to adjust the number of particles will have been added to the program. However, it is possible in Version 5 for multiple particles to inhabit the same point of space, a physically unrealistic phenomenon. In order to correct this, the possible directions for the particles must be adjusted to allow a full range of motion. Furthermore, an algorithm will be incorporated in to dictate the collisions of the particles.
	\item[Version 7] This version will deal with non-parallelepiped regions. I will begin by studying the class of all convex, closed, and continuously differentiable surfaces. I am unsure, but my intuition tells me that all such surfaces are topologically equivalent. Thus, any such surface should be able to be created by morphing a sphere.
\end{description}

\end{document}
